\maketitle

\begin{abstract}
    This is a sample document for the \texttt{nankai} class.
    It demonstrates how to use various features of the class.
\end{abstract}

\tableofcontents

\clearpage

\section{Paragraphs}

\subsection{Normal Text}

\lipsum[1-20]

\subsection{Monospace Text}

\texttt{Hello, world!}

\subsection{List}

Here is a list of items:

\begin{itemize}
    \item First item
    \item Second item
    \item Third item
\end{itemize}

\subsection{Numbered List}

Here is a numbered list of items:

\begin{enumerate}
    \item First item
    \item Second item
    \item Third item
\end{enumerate}

\section{Document Components}

\subsection{Footnotes}

\lipsum[1]\footnote{This is a footnote.}

\lipsum[2]\footnote{This is another footnote.}

\subsection{Bibliography}

\lipsum[3]~\cite{SCSF00045714}

\section{Tables and Figures}

\begin{figure}[H]
    \centering
    \includegraphics[width=0.3\linewidth]{example-image-a}
    \caption{This is a figure.}
    \label{fig:example}
\end{figure}

\begin{table}[H]
    \centering
    \begin{tabular}{cc}
        \toprule
        \textbf{A} & \textbf{B} \\
        \midrule
        1          & 2          \\
        \bottomrule
    \end{tabular}
    \caption{This is a table.}
    \label{tab:example}
\end{table}

\section{Notes and Crumbs}

\begin{note}
    This is a note.
\end{note}

Here is a crumb:

\PrintCrumb{Crumb Title}{Crumb content goes here.}

\PrintBibliography
